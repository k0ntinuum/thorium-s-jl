

\documentclass{article}
\usepackage[utf8]{inputenc}
\usepackage{setspace}
\usepackage{ mathrsfs }
\usepackage{graphicx}
\usepackage{amssymb} %maths
\usepackage{amsmath} %maths
\usepackage[margin=0.2in]{geometry}
\usepackage{graphicx}
\usepackage{ulem}
\setlength{\parindent}{0pt}
\setlength{\parskip}{10pt}
\usepackage{hyperref}
\usepackage[autostyle]{csquotes}

\usepackage{cancel}
\renewcommand{\i}{\textit}
\renewcommand{\b}{\textbf}
\newcommand{\q}{\enquote}
%\vskip1.0in





\begin{document}

{\setstretch{0.0}{


\b{Thorium} is a specialized binary version of Cesium. It uses $n \times n$  matrix of bits as a key. The trace of the matrix is added (mod 2) to the plaintext bit to get the ciphertext bit. The key matrix is then \q{shredded} as a function of its current diagonal and the processed plaintext bit.


Here's the encoding function: 


\begin{verbatim}
function encode(p,q)
    k = copy(q)
    c = Bool[]
    for i in eachindex(p)
        push!(c,Bool((tr(k) + p[i])%2))
        k = spin(k,p[i])
    end
    c
end

\end{verbatim}

It's crucial that the key evolve as a function of the plaintext, or we could have just used a simple masking bit-vector to begin with. Here's the update of the key. Note that Julia gives us a convenient circular shift of rows or columns as needed. 

\begin{verbatim}
spin_row(k,i,p) = circshift(k[i,:], k[i,i] + p + 1)

spin_col(k,i,p) = circshift(k[:,i], k[i,i] + p + 1)

function spin(q,p)
    k = copy(q)
    for j in 1:n[begin]
        if  Bool((j+p)%2) 
            k[j,:] = spin_row(k, j, p)
        else 
            k[:,j] = spin_col(k, j, p)
        end
    end
    k
end

\end{verbatim}

You can see that $n$ spins are performed, alternating between rows or columns. The bit $p$ determines whether rows or columns go first, \b{and} this bit boosts (or not) every such spin. 


For encryption there's also an \q{autospin} feature that prepares a different evolution of the key for each round. The columns of the key are fed like $n$ plaintext bits to guide this evolution.

\begin{verbatim}

function auto_spin(q,c)
    k = copy(q)
    for i in 1:n  k = spin(k,k[c,i]) end
    k
end

\end{verbatim}


}}
\end{document}
